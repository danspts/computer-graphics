\documentclass{article}

\usepackage[a4paper,hmargin=2cm,vmargin=2cm]{geometry}
\usepackage{amsmath,amssymb,amsthm}
\usepackage{layout}
\usepackage{listings}
\usepackage{tikz}
\usepackage{graphicx,verbatim,xspace,color, subcaption}
\usepackage{fullpage}
\usepackage{float}

\floatstyle{boxed} \restylefloat{figure}

\setlength{\oddsidemargin}{0pt}

\begin{document}
    \title{Exercise 2}
    \author{227367455}
    \date{}
    \maketitle

    \section*{Question 1}
    Let $\ell_1$ be the line that passes through $p_1=(2,9,8)$ and $p_2=(1,9,9)$ 
    and let $\ell_2$ be the line that passes through $p_3=(1,1,1)$ and $p_4=(2,5,4)$
    \begin{enumerate}
        \item Find out if the two lines intersect and if so, find the intersection point of $\ell_1$ and $\ell_2$
        
        $u_1 = p_2 - p_1 = \begin{bmatrix} 1 - 2 \\ 9 - 9 \\ 9 - 8 \end{bmatrix}$
        $ = \begin{bmatrix} -1 \\ 0 \\ 1 \end{bmatrix}$

        $\ell_1(\lambda) = p_1 + \lambda u_1 = \begin{bmatrix} 1 \\ 9 \\ 9 \end{bmatrix} + \lambda \begin{bmatrix} -1 \\ 0 \\ 1 \end{bmatrix}$

        $u_2 = p_4 - p_3 = \begin{bmatrix} 2 - 1 \\ 5 - 1 \\ 4 - 1 \end{bmatrix}$
        $ = \begin{bmatrix} 1 \\ 4 \\ 3 \end{bmatrix}$

        $\ell_2(\lambda) = p_3 + \lambda u_2 = \begin{bmatrix} 1 \\ 1 \\ 1 \end{bmatrix} + \lambda \begin{bmatrix} 1 \\ 4 \\ 3 \end{bmatrix}$

        $\frac{u_1 \cdot u_2}{\Vert u_1 \Vert \Vert u_2 \Vert} = \frac{-1 + 3}{\sqrt{2} \sqrt{1 + 16 + 9}} \neq 1, -1$
        $\implies \ell_1$ is not parallel to $\ell_2 $ 

        $\begin{bmatrix} 1 \\ 9 \\ 9 \end{bmatrix} + \lambda_1 \begin{bmatrix} -1 \\ 0 \\ 1 \end{bmatrix} = $
        $\begin{bmatrix} 1 \\ 1 \\ 1 \end{bmatrix} + \lambda_2 \begin{bmatrix} 1 \\ 4 \\ 3 \end{bmatrix}$

        $\iff \lambda_2 \begin{bmatrix} 1 \\ 4 \\ 3 \end{bmatrix} + \lambda_1 \begin{bmatrix} 1 \\ 0 \\ -1 \end{bmatrix} = \begin{bmatrix} 0 \\ 8 \\ 8 \end{bmatrix} $

        $\iff \lambda_2 = -\lambda_1 = 2$

        Now we plug in!

        $p_0 = \ell_2(2) = \begin{bmatrix} 1 \\ 1 \\ 1 \end{bmatrix} + (2) \begin{bmatrix} 1 \\ 4 \\ 3 \end{bmatrix}$
        $=(1 + 2(1), 1 + 2(4), 1 + 2(3)) = (3, 9, 7)$ $\square$

        \item Let $S$ be the sphere whose center is the intersection of $\ell_1$ and $\ell_2$ and whose radius is $r=4$.
         Write the implicit representation of the sphere.

        $(x - 3)^2 + (y - 9)^2 + (z - 7)^2 = 4^2$ 

        \item Find the implicit representation of the two planes $\Pi_1$ and $\Pi_2$ that are tangent to the sphere $S$ at the points of intersections of $\ell_1$ towards $p_2$ and $\ell_2$ towards $p_4$, respectively.
        
        $(1 - \lambda - 3)^2 + (9 - 9)^2 + (9 + \lambda - 7)^2 = 4^2$

        $(-2 - \lambda)^2 + (2 + \lambda)^2 = 16$

        $(2 + \lambda)^2 = 8$

        $2 + \lambda = \pm \sqrt{8}$

        $\lambda = \pm \sqrt{8} - 2$

        We get $(3 \mp \sqrt{8}, 9, 7 \pm \sqrt{8})$, using euclidian distance, we get $(3 - \sqrt{8}, 9, 7 + \sqrt{8})$

        Now since the line are passing through the center then they are normal to the sphere at their point of intersection

        We do $(3 - \sqrt{8}, 9, 7 + \sqrt{8}) \cdot (-1, 0, 1) = 7 - 3 + 2 \sqrt{8} = 4 + 2 \sqrt{8}$

        We get the plane equation $\Pi_1 = (-1, 0, 1) \cdot (x, y, z) - 4 - 2 \sqrt{8} = 0$

        Since we get a negative value when we plug in $p_2$ we negate the normal and obtain:

        $\Pi_1 = (1, 0, -1) \cdot (x, y, z) + 4 + 2 \sqrt{8} = 0$ \\

        $(1 + \lambda - 3)^2 + (1 + 4 \lambda - 9)^2 + (1 + 3 \lambda - 7)^2 = 4^2$

        $(\lambda - 2)^2 + (-8 + 4\lambda)^2 + (-6 + 3 \lambda)^2 = 16$

        $4 - 4 \lambda + \lambda^2 + 64 - 64\lambda + 16 \lambda^2 + 36 - 36\lambda + 9 \lambda^2 = 16$

        $26 \lambda^2 - 104 \lambda + 88 = 0$

        Solving the equation: $\lambda = 2.7845, 1.2155 $
        
        We repeat the procedure as above:

        $(3.7845 , 12.138, 9.3535)$, $(2.2155, 4.862, 3.6465)$ and take $(2.2155, 4.862, 3.6465)$ as its distance is the closest to $p_4$

        $(2.2155, 4.862, 3.6465) \cdot (1,4,3) = 32.603$

        We get the plane equation $\Pi_1 = (1, 4, 3) \cdot (x, y, z) - 32.6 = 0$

        If we plug in $p_4$ we get a positive value so our surface is in the right direction $\square$

    \end{enumerate}
    \section*{Question 2}

    \begin{enumerate}
        \item Let $\ell=(2,0,3)+ \lambda(1,5,2)$ be a line. Find the projection of $p=(3,-1,4)$ on $\ell$

        Clearly $p_0 = \ell(0) = (2,0,3)$ is a point on $\ell$ 

        We will project $p - p_0$ onto $\ell$

        Let $\hat{u} = \frac{\begin{bmatrix} 1 \\ 5 \\ 2\end{bmatrix}}
        { \Bigg \| \begin{bmatrix} 1 \\ 5 \\ 2\end{bmatrix} \Bigg \| }$
        $ = \frac{1}{ \sqrt{1^2 + 5^2 + 2^2} }  \begin{bmatrix} 1 \\ 5 \\ 2 \end{bmatrix}$
        $ = \frac{1}{ \sqrt{30} }\begin{bmatrix} 1 \\ 5 \\ 2\end{bmatrix}$

        $proj_{\hat{u}}(p - p_0)= (\hat{u} \cdot (p - p_0)) \hat{u} $

        $= \frac{1}{30} \Bigg (\begin{bmatrix} 1 \\ 5 \\ 2\end{bmatrix} \cdot \begin{bmatrix} 3 - 2 \\ -1 - 0 \\ 4 - 3 \end{bmatrix} \Bigg )\begin{bmatrix} 1 \\ 5 \\ 2\end{bmatrix}$

        $= \frac{1}{30} \Bigg (\begin{bmatrix} 1 \\ 5 \\ 2\end{bmatrix} \cdot \begin{bmatrix} 1 \\ -1 \\ 1 \end{bmatrix} \Bigg )\begin{bmatrix} 1 \\ 5 \\ 2\end{bmatrix}$

        $= \frac{1}{30} (1 - 5 + 2) \begin{bmatrix} 1 \\ 5 \\ 2\end{bmatrix}$

        $= \frac{1}{30} (-2) \begin{bmatrix} 1 \\ 5 \\ 2\end{bmatrix}$

        $= \frac{-1}{15} \begin{bmatrix} 1 \\ 5 \\ 2\end{bmatrix}$

        Now to find the projection of the point p we do:

        $p_0 + proj_{\hat{u}}(p - p_0) = (2, 0, 3) + \frac{-1}{15} \begin{bmatrix} 1 \\ 5 \\ 2\end{bmatrix}$

        $ = (1 \frac{14}{15}, -\frac{1}{3}, 2 \frac{13}{15} )$ $\square$

        \item Find an implicit and a parametric representation for the plane that contains both p and $\ell$
       
        $v = (p - p_0) =  \begin{bmatrix} 3 - 2 \\ -1 - 0 \\ 4 - 3 \end{bmatrix} = \begin{bmatrix} 1 \\ -1 \\ 1 \end{bmatrix}$

        $\Pi(\lambda, \mu) = (2,0,3) + \lambda \begin{bmatrix} 1 \\ 5 \\ 2\end{bmatrix} + \mu \begin{bmatrix} 1 \\ -1 \\ 1 \end{bmatrix}$

        $\begin{bmatrix} 1 \\ -1 \\ 1 \end{bmatrix} \times \begin{bmatrix} 1 \\ 5 \\ 2\end{bmatrix} = \Bigg | \begin{matrix}
            i & j & k \\
            1 & 5 & 2 \\
            1 & -1 & 1
        \end{matrix} \Bigg | = [(5)(1) - (2)(-1)] i - [(1)(1) - (1)(2)] j + [(1)(-1) - (5)(1)] k$
        $ = 7i + j - 6k$

        $n = \begin{bmatrix} 7 \\ 1 \\ -6 \end{bmatrix}$

        $n \cdot ((2, 0, 3) - (0, 0, 0)) = \begin{bmatrix} 7 \\ 1 \\ -6 \end{bmatrix} \cdot \begin{bmatrix} 2 \\ 0 \\ 3 \end{bmatrix}$
        $ = 14 - 18 = -4 $

        $\Pi : \begin{bmatrix} 7 \\ 1 \\ -6 \end{bmatrix} \cdot \begin{bmatrix} x \\ y \\ z \end{bmatrix} = -4$ $\square$
        
    \end{enumerate}
    
    \section*{Question 3}
    Given the following points: 
    
    $P_0 =(-1,-1,3),P_1 =(-1,3,-1),P_2=(3,3,5),P_3=(3,5,3),P_4=(\frac{-1}{2},3,3)$

    \begin{enumerate}
        \item Show that $P_0, P_1, P_2, P_3 $ lie on the same plane, $H$, and find the implicit equation of $H$

        $P_1 - P_0 = \begin{bmatrix} -1 - (-1) \\ 3 - (-1) \\ -1 - 3 \end{bmatrix} = \begin{bmatrix} 0 \\ 4 \\ -4 \end{bmatrix}$

        $P_1 - P_0 = \begin{bmatrix} 3 - (-1) \\ 3 - (-1) \\ 5 - 3 \end{bmatrix} = \begin{bmatrix} 4 \\ 4 \\ 2 \end{bmatrix}$

        $ \Bigg | \begin{matrix}
            i & j & k \\
            0 & 4 & -4 \\
            4 & 4 & 2
        \end{matrix} \Bigg |$
        $ = [(4)(2) - (-4)(4)]i - [(0)(2) - (-4)(4)]j + [(0)(4) - (4)(4)]$
        $ = 24i - 16j -16k = 3i - 2j - 2k$

        $n = \begin{bmatrix} 3 \\ -2 \\ -2 \end{bmatrix}$

        $\begin{bmatrix} 3 \\ -2 \\ -2 \end{bmatrix} \cdot \begin{bmatrix} -1 \\ -1 \\ 3 \end{bmatrix} $
        $ = -3 + 2 - 6 = -7$

        Therefore we can build H out of these 3 points.

        $H : \begin{bmatrix} 3 \\ -2 \\ -2 \end{bmatrix} \cdot \begin{bmatrix} x \\ y \\ z \end{bmatrix} + 7 = 0$

        Now plug in $P_3$:

        $\begin{bmatrix} 3 \\ -2 \\ -2 \end{bmatrix} \cdot \begin{bmatrix} 3 \\ 5 \\ 3 \end{bmatrix} + 7 $
        $ = 9 - 10 - 6 + 7 = 0$ as needed
        
        Therefore since $P_3$ lies on $H$ which was made with $P_0$, $P_1$ and $P_2$, then they all belong to the same plane H

        \item b) Determine the outwards facing unit normal vector of each triangular face Calculate the implicit representation of the planes
        
        In order to do so we can iteratively find the normal n and invert it if we get a positive value when plugging in the 2 remaining points.
        However since the picture is provided we will use the RHR.        
        
        



        \paragraph{} $H_1$ defined by $P_0,P_2,P_4$:

        $v_0 = 2(P_4 - P_0) = \begin{bmatrix} 1 \\ 8 \\ 0 \end{bmatrix}$,
        $v_1 = P_2 - P_0 = \begin{bmatrix} 4  \\ 4 \\ 2 \end{bmatrix}$,

        $ v_1 \times v_0 = \Bigg | \begin{matrix}
            i & j & k \\
            4  & 4 & 2 \\
            1 & 8 & 0 \\
        \end{matrix} \Bigg | = \begin{bmatrix}  -16 \\ 2 \\ 28 \end{bmatrix} = \begin{bmatrix}  - 8 \\ 1 \\ 14 \end{bmatrix} = \begin{bmatrix}  - \frac{8}{261} \\ \frac{1}{261} \\ \frac{14}{261} \end{bmatrix}$

        $d = ()$
        \paragraph{} $H_2$ defined by $P_0,P_1,P_4$:

        $v_0 = P_1 - P_0 = \begin{bmatrix} 0 \\ 4 \\ -4 \end{bmatrix}$,
        $v_1 = 2(P_4 - P_0) = \begin{bmatrix} 1  \\ 8 \\ 0 \end{bmatrix}$,

        $ v_0 \times v_1  = \Bigg | \begin{matrix}
            i & j & k \\
            1 & 8 & 0\\
            0 & 4 & -4 

        \end{matrix} \Bigg | = \begin{bmatrix} -32 \\ 4 \\ 4 \end{bmatrix} = \begin{bmatrix} -8 \\ 1 \\ 1 \end{bmatrix}  = \begin{bmatrix} \frac{-8}{66} \\ \frac{1}{66} \\ \frac{1}{66} \end{bmatrix}$

        \paragraph{} $H_3$ defined by $P_1,P_3,P_4$ :

        $v_0 = P_3 - P_1 = \begin{bmatrix} 4 \\ 2 \\ 4 \end{bmatrix}$,
        $v_1 = 2(P_4 - P_1) = \begin{bmatrix} 1  \\ 0 \\ 8 \end{bmatrix}$,

        $ v_1 \times  v_0  = \Bigg | \begin{matrix}
            i & j & k \\
            1 & 0 & 8 \\
            4 & 2 & 4
        \end{matrix} \Bigg | = \begin{bmatrix} -16 \\ 28 \\ 2 \end{bmatrix} = \begin{bmatrix} -8 \\ 14 \\ 1 \end{bmatrix}  = \begin{bmatrix} \frac{-8}{261} \\ \frac{14}{261} \\ \frac{1}{261}  \end{bmatrix} $

        \paragraph{} $H_4$ defined by $P_2,P_3,P_4$:

        $v_0 = P_2 - P_3 = \begin{bmatrix} 0 \\ 2 \\ -2 \end{bmatrix}$,
        $v_1 = 2(P_4 - P_3) = \begin{bmatrix} -7  \\ -4 \\ 0 \end{bmatrix}$,

        $ v_1 \times v_0  = \Bigg | \begin{matrix}
            i & j & k \\
            -7 & -4 & 0 \\
            0 & -2 & 2
        \end{matrix} \Bigg | = \begin{bmatrix} -8 \\ 14 \\ 14 \end{bmatrix} = \begin{bmatrix} -4 \\ 7 \\ 7 \end{bmatrix}  = \begin{bmatrix} \frac{-4}{114} \\ \frac{7}{114} \\ \frac{7}{114} \end{bmatrix} $

        \item c)
        
        $(- 8, 1, 14) \cdot  (\frac{-1}{2},3,3) = 4 + 3 + 42 = 49$

        $H_1 : - 8x + y + 14z = 49$

        $(-8, 1, 1)  \cdot (\frac{-1}{2},3,3) = 4 + 3 + 3 = 10$

        $H_2 : - 8x + y + z = 10$

        $(-8, 14, 1) \cdot (\frac{-1}{2},3,3) = 4 + 42 + 3 = 49 $

        $H_3 : - 8x + 14y + 1 = 49$

        $(-4, 7, 7) \cdot (\frac{-1}{2},3,3) = 4 + 21 + 21 = 46$

        $H_4 : - 4x + 7y + 7z = 46$

        And we already calculated H

        \item d)
        We plug in in all the shapes and it must be negative for all shapes except H
        i)
        Is inside because :
        $H(-1/2, 1, 2) > 0$ and  $H1(-1/2, 1, 2),  H2(-1/2, 1, 2),  H3(-1/2, 1, 2),  H4(-1/2, 1, 2) < 0$
        ii)
        Outside
        $H(1, 0, 1) < 0 $
        iii)
        Outside 
        $H(3, 2, 4) < 0 $

    \end{enumerate}

    \section*{Question 4}
    A 2D light ray is sent from point P=(1,-1). It is reflected off a surface (represented by a line) at $R=(6,11)$, and reaches a receiver point at $Q=(25,13\frac{2}{17})$.
     Note that light rays hitting a surface reflect in a direction which is symmetric according to the normal.
    \begin{enumerate}
        \item Find the implicit representation of the surface s.t. its “up” is towards P (i.e. it faces the incoming ray). See the illustration below. The small black arrow is the direction the surface is facing.
        $RP = P - R = (5, 12), RQ = Q - R = (19, \frac{36}{17})$
        
        Normalizing we get $\hat{RP} = \frac{(5,12)}{\sqrt{5^2 + 12^2}} = \frac{(5,12)}{13}$

        Normalizing we get $\hat{RQ} = \frac{(19, \frac{36}{17})}{\sqrt{19^2 + \frac{36^2}{17^2}}} = \frac{(19, \frac{36}{17})}{\sqrt{\frac{36^2 + 323^2}{17^2}}} = \frac{(19, \frac{36}{17})}{\frac{325}{17}}$

        Now we want to get  $RQ - RP$ (Think of a diamond shaped figure, we could also get the average) 

        $n = (\frac{ 17 \cdot 19}{325} - \frac{5}{13},\frac{36}{325}  - \frac{5}{13}) = (\frac{198}{325} , \frac{-264}{325}) =  (3 , -4)$

        $n \cdot (R - 0) = 6 \cdot 3 + 11 (-4) = 18 - 44 = -26$

        We get $(3, -4) \cdot (x, y) + 26 = 0$

        \item Find the angle between the ray and the surface.
        
        $ cos(\alpha) = \frac{n \cdot RP }{ \| n \| \cdot \|RP \|} = \frac{ 3(5) + (-4)12 }{13 \cdot \sqrt{3^2 + 4^2}} = \frac{-33}{13 \cdot 5} =\frac{-33}{65}  $

        $\theta =   \alpha - \frac{\pi}{2} =  \arccos(\frac{-1}{13}) - \frac{\pi}{2} = 120.51^o - 90^o = 30.51^o$
    \end{enumerate}
\end{document}