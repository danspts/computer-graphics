\documentclass{article}

\usepackage[a4paper,hmargin=2cm,vmargin=2cm]{geometry}
\usepackage{amsmath,amssymb,amsthm}
\usepackage{layout}
\usepackage{listings}
\usepackage{tikz}
\usepackage{graphicx,verbatim,xspace,color, subcaption}
\usepackage{fullpage}
\usepackage{float}

\floatstyle{boxed} \restylefloat{figure}

\setlength{\oddsidemargin}{0pt}

\begin{document}
    \title{Exercise 2}
    \author{227367455}
    \date{}
    \maketitle

    \section*{Question 1}
    Let $\ell_1$ be the line that passes through $p_1=(2,9,8)$ and $p_2=(1,9,9)$ 
    and let $\ell_2$ be the line that passes through $p_3=(1,1,1)$ and $p_4=(2,5,4)$
    \begin{enumerate}
        \item Find out if the two lines intersect and if so, find the intersection point of $\ell_1$ and $\ell_2$
        
        $u_1 = p_2 - p_1 = \begin{bmatrix} 1 - 2 \\ 9 - 9 \\ 9 - 8 \end{bmatrix}$
        $ = \begin{bmatrix} -1 \\ 0 \\ 1 \end{bmatrix}$

        $\ell_1(\lambda) = p_1 + \lambda u_1 = \begin{bmatrix} 1 \\ 9 \\ 9 \end{bmatrix} + \lambda \begin{bmatrix} -1 \\ 0 \\ 1 \end{bmatrix}$

        $u_2 = p_4 - p_3 = \begin{bmatrix} 2 - 1 \\ 5 - 1 \\ 4 - 1 \end{bmatrix}$
        $ = \begin{bmatrix} 1 \\ 4 \\ 3 \end{bmatrix}$

        $\ell_2(\lambda) = p_3 + \lambda u_2 = \begin{bmatrix} 1 \\ 1 \\ 1 \end{bmatrix} + \lambda \begin{bmatrix} 1 \\ 4 \\ 3 \end{bmatrix}$

        $\frac{u_1 \cdot u_2}{\Vert u_1 \Vert \Vert u_2 \Vert} = \frac{-1 + 3}{\sqrt{2} \sqrt{1 + 16 + 9}} \neq 1, -1$
        $\implies \ell_1$ is not parallel to $\ell_2 $ 

        $\begin{bmatrix} 1 \\ 9 \\ 9 \end{bmatrix} + \lambda_1 \begin{bmatrix} -1 \\ 0 \\ 1 \end{bmatrix} = $
        $\begin{bmatrix} 1 \\ 1 \\ 1 \end{bmatrix} + \lambda_2 \begin{bmatrix} 1 \\ 4 \\ 3 \end{bmatrix}$

        $\iff \lambda_2 \begin{bmatrix} 1 \\ 4 \\ 3 \end{bmatrix} + \lambda_1 \begin{bmatrix} 1 \\ 0 \\ -1 \end{bmatrix} = \begin{bmatrix} 0 \\ 8 \\ 8 \end{bmatrix} $

        $\iff \lambda_2 = -\lambda_1 = 2$

        Now we plug in!

        $p_0 = \ell_2(2) = \begin{bmatrix} 1 \\ 1 \\ 1 \end{bmatrix} + (2) \begin{bmatrix} 1 \\ 4 \\ 3 \end{bmatrix}$
        $=(1 + 2(1), 1 + 2(4), 1 + 2(3)) = (3, 9, 7)$

        \item Let $S$ be the sphere whose center is the intersection of $\ell_1$ and $\ell_2$ and whose radius is $r=4$.
         Write the implicit representation of the sphere.

        $(x - 3)^2 + (y - 9)^2 + (z - 7)^2 = 4^2$

        \item Find the implicit representation of the two planes $\Pi_1$ and $\Pi_2$ that are tangent to the sphere $S$ at the points of intersections of $\ell_1$ towards $p_2$ and $\ell_2$ towards $p_4$, respectively.
        
        $(1 - \lambda - 3)^2 + (9 - 9)^2 + (9 + \lambda - 7)^2 = 4^2$

        $(-2 - \lambda)^2 + (2 + \lambda)^2 = 16$

        $(2 + \lambda)^2 = 8$

        $2 + \lambda = \pm \sqrt{8}$

        $\lambda = \pm \sqrt{8} - 2$


        $(1 + \lambda - 3)^2 + (1 + 4 \lambda - 9)^2 + (1 + 3 \lambda - 7)^2 = 4^2$

        $(\lambda - 2)^2 + (-8 + 4\lambda)^2 + (-6 + 3 \lambda)^2 = 16$

        $4 - 4 \lambda + \lambda^2 + 64 - 64\lambda + 16 \lambda^2 + 36 - 36\lambda + 9 \lambda^2 = 16$

        $26 \lambda^2 - 104 \lambda + 88 = 0$



    \end{enumerate}
    \section*{Question 2}

    \begin{enumerate}
        \item Let $\ell=(2,0,3)+ \lambda(1,5,2)$ be a line. Find the projection of $p=(3,-1,4)$ on $\ell$

        Clearly $p_0 = \ell(0) = (2,0,3)$ is a point on $\ell$ 

        We will project $p - p_0$ onto $\ell$

        Let $\hat{u} = \frac{\begin{bmatrix} 1 \\ 5 \\ 2\end{bmatrix}}
        { \Bigg \| \begin{bmatrix} 1 \\ 5 \\ 2\end{bmatrix} \Bigg \| }$
        $ = \frac{1}{ \sqrt{1^2 + 5^2 + 2^2} }  \begin{bmatrix} 1 \\ 5 \\ 2 \end{bmatrix}$
        $ = \frac{1}{ \sqrt{30} }\begin{bmatrix} 1 \\ 5 \\ 2\end{bmatrix}$

        $proj_{\hat{u}}(p - p_0)= (\hat{u} \cdot (p - p_0)) \hat{u} $

        $= \frac{1}{30} \Bigg (\begin{bmatrix} 1 \\ 5 \\ 2\end{bmatrix} \cdot \begin{bmatrix} 3 - 2 \\ -1 - 0 \\ 4 - 3 \end{bmatrix} \Bigg )\begin{bmatrix} 1 \\ 5 \\ 2\end{bmatrix}$

        $= \frac{1}{30} \Bigg (\begin{bmatrix} 1 \\ 5 \\ 2\end{bmatrix} \cdot \begin{bmatrix} 1 \\ -1 \\ 1 \end{bmatrix} \Bigg )\begin{bmatrix} 1 \\ 5 \\ 2\end{bmatrix}$

        $= \frac{1}{30} (1 - 5 + 2) \begin{bmatrix} 1 \\ 5 \\ 2\end{bmatrix}$

        $= \frac{1}{30} (-2) \begin{bmatrix} 1 \\ 5 \\ 2\end{bmatrix}$

        $= \frac{-1}{15} \begin{bmatrix} 1 \\ 5 \\ 2\end{bmatrix}$

        Now to find the projection of the point p we do:

        $p_0 + proj_{\hat{u}}(p - p_0) = (2, 0, 3) + \frac{-1}{15} \begin{bmatrix} 1 \\ 5 \\ 2\end{bmatrix}$

        $ = (1 \frac{14}{15}, -\frac{1}{3}, 2 \frac{13}{15} )$

        \item Find an implicit and a parametric representation for the plane that contains both p and $\ell$
       
        $v = (p - p_0) =  \begin{bmatrix} 3 - 2 \\ -1 - 0 \\ 4 - 3 \end{bmatrix} = \begin{bmatrix} 1 \\ -1 \\ 1 \end{bmatrix}$

        $\Pi(\lambda, \mu) = (2,0,3) + \lambda \begin{bmatrix} 1 \\ 5 \\ 2\end{bmatrix} + \mu \begin{bmatrix} 1 \\ -1 \\ 1 \end{bmatrix}$

        $\begin{bmatrix} 1 \\ -1 \\ 1 \end{bmatrix} \times \begin{bmatrix} 1 \\ 5 \\ 2\end{bmatrix} = \Bigg | \begin{matrix}
            i & j & k \\
            1 & 5 & 2 \\
            1 & -1 & 1
        \end{matrix} \Bigg | = [(5)(1) - (2)(-1)] i - [(1)(1) - (1)(2)] j + [(1)(-1) - (5)(1)] k$
        $ = 7i + j - 6k$

        $n = \begin{bmatrix} 7 \\ 1 \\ -6 \end{bmatrix}$

        $n \cdot ((2, 0, 3) - (0, 0, 0)) = \begin{bmatrix} 7 \\ 1 \\ -6 \end{bmatrix} \cdot \begin{bmatrix} 2 \\ 0 \\ 3 \end{bmatrix}$
        $ = 14 - 18 = -4 $

        $\Pi : \begin{bmatrix} 7 \\ 1 \\ -6 \end{bmatrix} \cdot \begin{bmatrix} x \\ y \\ z \end{bmatrix} = -4$
        
    \end{enumerate}
    
    \section*{Question 3}
    Given the following points: 
    
    $P_0 =(-1,-1,3),P_1 =(-1,3,-1),P_2=(3,3,5),P_3=(3,5,3),P_4=(\frac{-1}{2},3,3)$

    \begin{enumerate}
        \item Show that $P_0, P_1, P_2, P_3 $ lie on the same plane, $H$, and find the implicit equation of $H$

        $P_1 - P_0 = \begin{bmatrix} -1 - (-1) \\ 3 - (-1) \\ -1 - 3 \end{bmatrix} = \begin{bmatrix} 0 \\ 4 \\ -4 \end{bmatrix}$

        $P_1 - P_0 = \begin{bmatrix} 3 - (-1) \\ 3 - (-1) \\ 5 - 3 \end{bmatrix} = \begin{bmatrix} 4 \\ 4 \\ 2 \end{bmatrix}$

        $ \Bigg | \begin{matrix}
            i & j & k \\
            0 & 4 & -4 \\
            4 & 4 & 2
        \end{matrix} \Bigg |$
        $ = [(4)(2) - (-4)(4)]i - [(0)(2) - (-4)(4)]j + [(0)(4) - (4)(4)]$
        $ = 24i - 16j -16k = 3i - 2j - 2k$

        $n = \begin{bmatrix} 3 \\ -2 \\ -2 \end{bmatrix}$

        $\begin{bmatrix} 3 \\ -2 \\ -2 \end{bmatrix} \cdot \begin{bmatrix} -1 \\ -1 \\ 3 \end{bmatrix} $
        $ = -3 + 2 - 6 = -7$

        Therefore we can build H out of these 3 points.

        $H : \begin{bmatrix} 3 \\ -2 \\ -2 \end{bmatrix} \cdot \begin{bmatrix} x \\ y \\ z \end{bmatrix} + 7 = 0$

        Now plug in $P_3$:

        $\begin{bmatrix} 3 \\ -2 \\ -2 \end{bmatrix} \cdot \begin{bmatrix} 3 \\ 5 \\ 3 \end{bmatrix} + 7 $
        $ = 9 - 10 - 6 + 7 = 0$ as needed
        
        Therefore since $P_3$ lies on $H$ which was made with $P_0$, $P_1$ and $P_2$, then they all belong to the same plane H

        \item b and c) Determine the outwards facing unit normal vector of each triangular face Calculate the implicit representation of the planes
        
        In order to do so we will iteratively find the normal n and invert it if we get a positive value when plugging in the 2 remaining points
        
        First we plug in $P_4$ in $H$

        $\begin{bmatrix} 3 \\ -2 \\ -2 \end{bmatrix} \cdot \begin{bmatrix} \frac{-1}{2} \\ 3 \\ 3 \end{bmatrix} + 7 < 7 - 12 < 0$ as needed


        $v_0 = 2(P_4 - P_0) = \begin{bmatrix} 1 \\ 8 \\ 0 \end{bmatrix}$,
        $v_1 = 2(P_4 - P_1) = \begin{bmatrix} 1  \\ 0 \\ 8 \end{bmatrix}$,

        $v_2 = 2(P_4 - P_2) = \begin{bmatrix} -7 \\ 0 \\ -4 \end{bmatrix}$,
        $v_3 = 2(P_4 - P_3) = \begin{bmatrix} -7 \\ -4 \\ 0 \end{bmatrix}$


        \paragraph{} $H_1$ defined by $P_0,P_2,P_4$ or $v_0, v_2, P_4$:

        $ \Bigg | \begin{matrix}
            i & j & k \\
            1 & 8 & 0 \\
            -7 & 0 & -4
        \end{matrix} \Bigg | = \begin{bmatrix}  - 32 \\ 4 \\ 56 \end{bmatrix} = \begin{bmatrix}  - 8 \\ 1 \\ 14 \end{bmatrix}$

        \paragraph{} $H_2$ defined by $P_0,P_1,P_4$ or $v_0, v_1, P_4$:

        $ \Bigg | \begin{matrix}
            i & j & k \\
            1 & 8 & 0 \\
            1 & 0 & 8
        \end{matrix} \Bigg | = \begin{bmatrix} 64 \\ -8 \\ -8 \end{bmatrix} = \begin{bmatrix} 16 \\ -2 \\ -2 \end{bmatrix}$

        \paragraph{} $H_3$ defined by $P_1,P_3,P_4$ or $v_0, v_2, P_4$:

        $ \Bigg | \begin{matrix}
            i & j & k \\
            1 & 0 & 8 \\
            -7 & -4 & 0
        \end{matrix} \Bigg | = \begin{bmatrix} 32 \\ -56 \\ -4 \end{bmatrix} = \begin{bmatrix}  8 \\ -14 \\ -1 \end{bmatrix}$

        \paragraph{} $H_4$ defined by $P_2,P_3,P_4$ or $v_3, v_2, P_4$:

        $ \Bigg | \begin{matrix}
            i & j & k \\
            -7 & 0 & -4 \\
            -7 & -4 & 0
        \end{matrix} \Bigg | = \begin{bmatrix} -16 \\ -28 \\ 28 \end{bmatrix} = \begin{bmatrix} -4 \\ -7 \\ 4 \end{bmatrix}$


        

    \end{enumerate}

    \section*{Question 4}
\end{document}